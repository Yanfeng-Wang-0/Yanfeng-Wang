\documentclass[11pt]{article}
\usepackage[a4paper,margin=1in]{geometry}
\usepackage{setspace}
\usepackage{lineno}
\usepackage{graphicx}
\usepackage{amsmath}
\usepackage{natbib}
\usepackage{hyperref}
\usepackage{pgfplotstable}
\usepackage{booktabs}
\usepackage{float}
\pgfplotsset{compat=newest}

% Set 1.5 spacing and continuous line numbers
\onehalfspacing
\linenumbers

\title{Comparison of Microbial Population Growth Models}
\author{Yanfeng Wang \\
Living Planet with Computational Methods in Ecology and Evolution\\
yw4524@ic.ac.uk}
\date{\textbf{Word Count: 2571}}

\begin{document}

\maketitle

\begin{abstract}
This project studies microbial population growth using various of models and aims to compare the extent of fitting between mathematical and phenomenological methods. The author uses seven different models, and applies with $R^2$, AIC, BIC for evaluation. The result shows GAM has the best outcome in most of conditions compare to others, and linear models such as exponential growth and power growth model have the worst performance.
\end{abstract}

\section{Introduction}
The data was provided for the project has multiple variables, and this study only discuss the relation between the measurement of population or biomass (Popbio) and time. Before constructing models, error data such as Popbio $<$ 0, Time $<$ 0 and NA term is removed. By Breunig(\cite{breunig2000lof}), cross validation can enhance the outlier identification when using the local outlier factor (LOF), hence outliers that are accepted by p-value test for all models and by LOF are removed. This study will use exponential growth model, power growth model, Logistic growth model, Gompertz growth model, generalized additive model (GAM), neutral network model and random forest model. The reason of choosing these specific models is that the Logistic and Gompertz model are suggested to be suitable for the growth of microorganisms, according to Verhulst and Winsor (\cite{verhulst1838logistic, winsor1932gompertz}). Also GAM, neutral network model and random forest model have excellent performance in processing nonlinear and high dimension data (\cite{hastie1986gam, mcculloch1943nn, breiman2001rf, wood2017gam}). Since there are 285 unique id number, the expected output is 285 files. By comparing the $R^2$, Akaike information criterion (AIC) and Bayesian information criterion (BIC) of each model for each unique id, the model with optimal score will be counted and accumulated. Eventually, three models with most optimal $R^2$, AIC, BIC will be discussed respectively.

\section{Methods}

\subsection{Data Processing}
During processing the data, local outlier factor (LOF) is used. LOF can measure the local density of each data point and identify if these points deviate significantly from their neighbors. Finally those points that are assumed to be outliers will be send to p-value test for each model. And they will be removed if and only if at least two models identify them are outliers.

\subsection{Model Evaluation Standards}

To determine the best fitting models, it is essential to apply with $R^2$, AIC, BIC. $R^2$ is a number less or equal to 1 that measures how well the model explain variance in the data. For $R^2$=1, it implies the model is perfectly fitted; for $R^2$=0, it implies that the model has no predictive ability; for $R^2$ $<$ 0, it indicates that the model is even worse than a simple prediction by taking the mean. Since $R^2$ can only measure the extent of model fitting but not the complexity of model. It is essential to introduce AIC and BIC. AIC evaluates the balance between the extent of model fitting and the complexity of model (is often represented by the number of coefficients in model) (\cite{akaike1974aic}). The lower AIC one model has, the better one model is. BIC has the same mechanism as AIC but with more strict to complexity. It penalizes models more for having excessive coefficients than AIC does. In general, AIC is suitable for small datasets, and BIC is suitable for large dataset (\cite{burnham2002model}).

\textbf{$R^2$:}

\begin{equation}
R^2 = 1 - \frac{SS_{res}}{SS_{tot}}
\end{equation}

where $SS_{res}$ is the residual sum of squares, i.e. sum of squared errors between predictions and actual values, and $SS_{tot}$ is the total sum of squares, i.e. sum of squared deviations between actual values and their mean.

\textbf{Akaike Information Criterion (AIC):}

\begin{equation}
AIC = 2k - 2 \ln(L)
\end{equation}

where $k$ is the number of model parameters and $L$ is the maximum likelihood estimation function.

\textbf{Bayesian Information Criterion (BIC):}

\begin{equation}
BIC = k \ln(n) - 2 \ln(L)
\end{equation}

where $k$ is the number of coefficients, $n$ is the number of observations and $L$ is the likelihood function.

\subsection{Mathematical Models}

By plot the graph of log(Popbio+1)~Time and log(log(Popbio+1)+1)~log(Time+1), it is shown that the curve between log(Popbio+1) and Time is similar to power growth and curve between log(log(Popbio+1)+1) and log(Time+1) is similar to exponential growth. Therefore it is reasonable to compute exponential growth model and power growth model. 

\textbf{Exponential Growth Model:}

\begin{equation}
\log(PopBio) = a \cdot Time + b
\end{equation}

where slope $a$ is growth rate, intercept $b$ is initial population.

\textbf{Power Growth Model:}

\begin{equation}
\log(PopBio) = a \cdot \log(Time) + b
\end{equation}

where $a$ and $b$ are coefficients determine scale and curvature.\\

\noindent Since the data is microbial population, Logistic growth model and Gompertz growth model as ordinary sigmoidal models can describe the growth of bacteria in medium. Hence these two will be used in this study as well.

\textbf{Logistic Growth Model:}

\begin{equation}
\log(PopBio) = \log\left(\frac{K}{1 + e^{-r (Time - t_0)}}\right)
\end{equation}

where $K$ is the maximum population size, i.e. carrying capacity, $r$ is the growth rate, $t_0$ is time shift that determines when growth begins.

\textbf{Gompertz Growth Model:}

\begin{equation}
\log(PopBio) = \log(K) - b \cdot e^{-r \cdot Time}
\end{equation}

where $K$ is the maximum population size, $b$ is the coefficient controls shape, $r$ is the growth rate.\\

\noindent By observing the plotted graph, several Unique IDs have very few data points. This may cause the mathematical models above are difficult to plot graphs, as they often require a sufficient number of points to achieve stable coefficient estimation and avoid overfitting (\cite{burnham2002model}). Therefore machine learning approaches are needed in this part . This can be attributed to the data augmentation using machine learning (\cite{bishop2006pattern}). The author chooses GAM, neutral network model and random forest model. GAM uses smooth functions which can avoid the problem from overfitting. Neutral network model can approximate arbitrary function as the population growth is unknown here, also the dataset is relatively large which is suitable for neutral network. Random forest model have good performance when data has missing value or the dataset is small, since it combines robustness into machine learning (\cite{lecun2015deep}).

\textbf{Generalized Additive Model (GAM):}

\begin{equation}
\log(PopBio) = \beta_0 + f(Time) + \epsilon
\end{equation}

where $f(Time)$ is a smooth function that can adapt to data.

\textbf{Neural Network Model:}

It will use nonlinear activation functions to model complex pattern in data, since without nonlinear activation functions it will only regress linearly.

\textbf{Random Forest Model:}

It will generate multiple decision trees to train from random subsets of data. Each tree will grow independently, and the result will be the mean of every tree.\\

\noindent In all seven models above, exponential growth and power growth model are linear, the rest are nonlinear. 

\section{Results}
For every output corresponds to an unique ID, $R^2$, AIC and BIC are calculted for each of 7 models. The best result (i.e. the highest $R^2$ and the lowest AIC and BIC) will be recorded and be accumulated into this list.

\begin{table}[H]
    \centering 
    \pgfplotstabletypeset[
        col sep=comma,
        string type,
        columns/Model/.style={string type},
        every head row/.style={before row=\toprule, after row=\midrule},
        every last row/.style={after row=\bottomrule}
    ]{../Results/model_comparison_summary.csv}
    \caption{Model Comparison Summary}
    \label{tab:model_comparison}
\end{table}

\noindent From the summary table above, the most $R^2$ 179 is obtained by GAM, and is followed by neutral network model and random forest model with 66 and 16 respectively. The lowest is 0 from power growth model. Exponential growth model and Gompertz model both have 1. And Logistic model is 2. For best AIC, GAM has the highest score 174. Power growth receives the lowest score 0. Random forest, neutral network, Logistic, Gompertz and exponential growth return 36, 22, 18, 10 and 5 respectively. For best BIC, GAM gets 158 which is the highest, while exponential growth, Gompertz, Logistic, power, neutral network and random forest is 4, 14, 24, 0, 17 and 48 respectively. This summary shows that GAM has best $R^2$, best AIC and best BIC, indicating it is the most suitable model comparing to all other six models in this dataset. Whereas power growth model receives the lowest in all $R^2$, AIC and BIC, which implies it is not suitable for this data compare to other models. Note that the expected output should have 285 files but only 265 are counted. This is because for unique numbers that have less than five data points are neglected, since fitting models to very few data will result in overfitting for complex models and instable coefficients for linear regression model.

\section{Discussion}

Since there are 265 graphs, 6 representative graphs are selected. First one is the graph with the most data points. We expect to see a complete growth cycle, in which models show complete process. The second is the graph with the least data points, 5 data points. We can see how models perform when lack of data. The third one is filtered with the condition, $R^2$ of linear models are worse than nonlinear models. With knowing the condition, we can analyze what data distribution cause this. The fourth one is selected by $R^2$ of linear models are better than nonlinear models. Also to interpret the reason of this from the data distribution. The fifth is the situation for both linear and nonlinear models perform well. And the last one is filtered by the best $R^2$, AIC and BIC that GAM generates. We can analyze the pattern of data points and why it is suitable for GAM.

\textbf{ID with the most data points}

This graph has 148 data points on it and time expands from 0 to 300. Observe that when data point is numerous and have relatively completion of growth period, linear model (exponential and power) do not have good performance. The reason is their coefficients are few and can not fit a complex shape. Logistic, Gompertz, neutral network and GAM all can fit the data to some extent. Random forest model can fits the data locally, however, the curve is not smooth enough due to the dependence of distribution of data point. 

\begin{figure}[H]
    \centering
    \includegraphics[width=1.0\textwidth]{../Results/Per_ID_Comparison/comparison_ID_282} 
    \caption{log(Popbio) vs Time Plot ID 282}
    \label{fig:scatter1}
\end{figure}

\noindent \textbf{One of ID with the fewest data points}

Observe that there are only five data points in the graph below. Under this condition, neutral network model and GAM both can fit the data perfectly, but the power growth curve shows contradictory trend. This can be attributed to the linear characteristic of power model and the power, a, of Time in power growth model. When a is small, model grow slow generally, hence cannot match the exponential growth stage. This will occur when the data point is too few to simulate the trend. And curves of Logistic and Gompertz are lines is because data points are too few for them to identify if they have enter the exponential stage.

\begin{figure}[H]
    \centering
    \includegraphics[width=0.9\textwidth]{../Results/Per_ID_Comparison/comparison_ID_106} 
    \caption{log(Popbio) vs Time Plot ID 106}
    \label{fig:scatter2}
\end{figure}

\noindent \textbf{Nonlinear models generally outperform than linear models}

By filtering the worst R square for linear models compare to nonlinear models, the graph below is generated. It can be observed that the failure of fitting model is attributed to the complexity of the curve, since the data points form a complete period of population growth which cannot be easily computed by linear models. Whereas, all nonlinear models perform better in general.This result has simliarity to the first graph above. We may assume that for large dataset with a completion growth period, nonlinear models outperform linear models.

\begin{figure}[H]
    \centering
    \includegraphics[width=0.9\textwidth]{../Results/Per_ID_Comparison/comparison_ID_7} 
    \caption{log(Popbio) vs Time Plot ID 7}
    \label{fig:scatter3}
\end{figure}

\noindent \textbf{Linear models generally outperform than nonlinear models}

Observe that data points in graph below array in a line, hence exponential model outperform nonlinear models. Although nonlinear models other than random forest also show line graphs and overlap together with exponential line, exponential model has lower AIC and BIC since it has few coefficients than others, which makes it become the best fit model in this situation. Random forest fail to fit the data because the data is too few for training for desicion trees.

\begin{figure}[H]
    \centering
    \includegraphics[width=0.9\textwidth]{../Results/Per_ID_Comparison/comparison_ID_220} 
    \caption{log(Popbio) vs Time Plot ID 220}
    \label{fig:scatter4}
\end{figure}

\noindent \textbf{Linear models perform good as well as nonlinear models}

Observe that this graph below has a sigmoidal shape, which implies this dataset has a relatively complete period, that is lag phase, exponential phase, stationary phase. This explains Logistic and Gompertz growth model have well fitted the data as they are typical models to describe this process. The reason that linear models have good performance as well as nonlinear models is the data is not plenty or dense enough to reflect the curve of the period. With few points in the turning area, linear models can have few residue sum of square, which contribute to $R^2$. 

\begin{figure}[H]
    \centering
    \includegraphics[width=0.9\textwidth]{../Results/Per_ID_Comparison/comparison_ID_158} 
    \caption{log(Popbio) vs Time Plot ID 158}
    \label{fig:scatter5}
\end{figure}

\noindent \textbf{GAM has the best performance among all}

In the graph below, GAM has the best performance among all models. Because data points distributed accord with the definition of GAM, which is computed with smooth functions. Also the smooth can bring advantage to AIC and BIC, hence other machine learning models cannot outperform than it.

\begin{figure}[H]
    \centering
    \includegraphics[width=0.9\textwidth]{../Results/Per_ID_Comparison/comparison_ID_94} 
    \caption{log(Popbio) vs Time Plot ID 94}
    \label{fig:scatter6}
\end{figure}

\noindent By observing but not limited to graphs above, noticeable severe overlaps always appear between neutral network model and GAM. However the summary table shows GAM has two to three times more of $R^2$ and eight times more of AIC and BIC than neutral network model. This can be attributed to the mechanism for neutral network model. When neutral network model training, it uses gradient descent(\cite{bishop1995neural}), hence for the function it generates usually is highly non-convex (\cite{goodfellow2016deep}), that is has multiple extrema. Therefore multiple locally minima occurs, and the optimization process will result in suboptimal solution. However, GAM uses smooth functions which has advantage to find global minimum (\cite{wood2017gam}). Thus although the plotted curves of GAM and neutral network model looks similar and overlaps highly. GAM generally has higher $R^2$ than neutral network model. Also since neutral network model uses many coefficients to do backpropagation which causes the imbalance between complexity of model and fitting well, the AIC and BIC of it is far below GAM (\cite{hastie2009elements}). Therefore GAM outperform neutral network model in all three standards. \\

\noindent The same question also occur between neutral network model and random forest model, i.e. neutral network model has outperformance than random forest on $R^2$, but poor performance on both AIC and BIC. Random forest takes mean on every decision trees it generates, hence more conservative on extreme patterns, which produce lower $R^2$. However, taking average can reduce overfitting and fluctuation in likelihood estimation function. Therefore, comparing to neutral network model who has plenty coefficients, random forest model has lower AIC and BIC.\\

\noindent In general, phenomenological models (i.e. GAM, neutral network model and random forest model) have better outcome compare to mathematical models (i.e. exponential growth, power growth, Logistic, Gompertz). The reason is the phenomenological models do not have fixed function form, which allow them to adapt patterns from the data rather than based on theoretical assumption. Also the biological growth contains noise data and irregular patterns, which can affect the accuracy of mathematical models, but do not harm machine learning models. However, the defect of phenomenological models is also noticeable, that is the interpretability is insufficient due to the lack of function.\\

\noindent Nonlinear models outperform linear models in most cases. Since microbial population growth graph often have sigmoidal shape, linear models with very few coefficients cannot compute such complex pattern. However, in some case like the graph of ID 220 above, when data points are few or array in a line, linear models can even outperform nonlinear models, since their AIC and BIC is lower. However this situation reflects the insufficiency of data, since it cannot perform complete growth period with lag phase, exponential phase, stationary phase and death phase. Therefore, the situation for linear models outperform nonlinear models is rare and irrational. \\

\noindent The choice of best model depends on various factors, including the number of data points, completeness of the dataset and the complexity of the microbial growth pattern. For large dataset with full growth cycle, GAM, Logistic model and neutral network model perform well (\cite{wood2017gam}), since these nonlinear models capture growth curve efficiently. For small dataset with few data points, using GAM and random forest model is efficient as GAM provides smooth estimation and random forest model avoids overfitting (\cite{breiman2001rf}). For nearly linear growth trend, simple linear models like exponential growth and power growth perform well at the meantime reduce the complexity. For highly complexity pattern, GAM and neutral network can balance the elasticity and extreme values. For limited transition phase in growth, Logistic and Gomopertz model can fit the curve well if enough data exists (\cite{zwietering1990bacterial}).

\section{Conclusion}
Overall, phenomenological models outperform mathematical models. Also nonlinear models have better performance than linear models. Generalized additive model (GAM) performs best among all other models with score 179/265 for $R^2$, 173/265 for AIC and 159/265 for BIC. While neutral network model has high $R^2$, it is penalized for complexity. Random forest model achieves better AIC and BIC from its decision tree mechanism. 

\bibliographystyle{apalike}
\bibliography{references}

\end{document}

